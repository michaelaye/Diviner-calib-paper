\section{Ground calibration campaign}

\subsection{Radiometric ground calibration}
Figure \ref{fig:gc_overview} shows the test configuration.  The blackbodies were originally built for calibration of the PMIRR instrument, and are scaled copies of the blackbodies used to calibrate ISAMS \cite{Nightingale_1991}. The back plate has a corrugated surface for increased effective emissivity. The inside of the blackbodies are painted with Nextel black paint (get details – what is the emissivity as a function of wavelength to 200 μm?). The two blackbodies were pointed roughly ±17.5$^\circ$ above and below the horizontal.  The DLRE instrument was positioned such that the blackbody apertures filled the two telescope apertures.  The fixed temperature blackbody was flooded with liquid nitrogen throughout the test.  The variable temperature blackbody varied in temperature from about 20 K to 415 K.
\begin{table}
\begin{tabular}{lll}
Parameter & Value & Unit\\
\hline
Detector Nominal IFOV Cross Track & 3.58 & mrad \\
Detector Nominal IFOV In Track & 6.15 & mrad \\
Nominal Telescope Aperture Diameter & 4 & cm\\
Reduction in Aperture Area & 6 & \% \\
Effective F number & 1.87 & \\
Etendue & 2.6E-04 & cm^{2}sr \\
Detector Dimension Cross Track & 0.024 & cm \\
Detector Dimension In Track & 0.048 & cm \\
Detector D^* & 8.0E+08 & cmHz^{1/2}/W \\
Signal Integration Time & 0.128 & s \\
Detector Noise-Equivalent Power (NEP) & 8.4E-11 & \\
\hline
\end{tabular}
\caption{\label{tab:parameters_shared} Parameters shared by all channels.}
\end{table}

