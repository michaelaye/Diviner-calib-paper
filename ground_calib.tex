\section{Ground calibration campaign}

\subsection{Radiometric ground calibration}
\subsubsection{Configuration}
Figure \ref{fig:gc_overview} shows the test configuration.
The blackbodies were originally built for calibration of the PMIRR instrument, and are scaled copies of the blackbodies used to calibrate ISAMS \cite{Nightingale_1991}. The back plate has a corrugated surface for increased effective emissivity.
The inside of the blackbodies are painted with Nextel black paint.
The two blackbodies were pointed roughly ±17.5$^\circ$ above and below the horizontal.
The DLRE instrument was positioned such that the blackbody apertures filled the two telescope apertures.
The fixed temperature blackbody was flooded with liquid nitrogen throughout the test.
The variable temperature blackbody varied in temperature from about 20 K to 415 K.

\subsubsection{Temperature sensors}
Two types of calibrated temperature sensors were used on the blackbodies.
The Rosemount 118MF2000 PRTs were calibrated at JPL in March 2002 from 91 to 350 K.
The silicon diode sensors are LakeShore DT-471-SD sensors, calibrated in February 2007 from 10 to 500 K.
Calibration data is at the end of this report and in attached documents.
The locations of the sensors are shown in Figures \ref{fig:bb_fixed_temp} and \ref{fig:bb_variable_temp}.
The LakeShore sensors were read out with a LakeShore 330 temperature controller, which had been recently calibrated.
The controller was set to use the standard LakeShore Curve 10 for all the sensors.
Thus, the raw silicon diode sensor temperatures need to be translated from Curve 10 to the specific calibration curve for each sensor.

\begin{table} 
    \begin{tabular}{ c c }
        a & b \\ 
        c & d \\ 
    \end{tabular} 
    \caption{Very important table} 
\end{table}



\begin{table} 
\begin{tabular}{lll}
Parameter & Value & Unit\\
\hline
Detector Nominal IFOV Cross Track & 3.58 & mrad \\
Detector Nominal IFOV In Track & 6.15 & mrad \\
Nominal Telescope Aperture Diameter & 4 & cm\\
Reduction in Aperture Area & 6 & \% \\
Effective F number & 1.87 & \\
Etendue & 2.6E-04 & cm$^2$sr \\
Detector Dimension Cross Track & 0.024 & cm \\
Detector Dimension In Track & 0.048 & cm \\
Detector $\mathrm{D^*}$ & 8.0E+08 & $\frac{cmHz^{1/2}}{W}$ \\
Signal Integration Time & 0.128 & s \\
Detector Noise-Equivalent Power (NEP) & 8.4E-11 & \\
\hline
\end{tabular}
\caption{\label{tab:parameters_shared} Parameters shared by all channels.}
\end{table}


