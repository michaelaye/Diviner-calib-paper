\section{General idea}
a.k.a. the elevator pitch:

\begin{enumerate}
\item In regular intervals, Diviner looks at space, to define the counts for zero radiation, defining a $counts_{space}$ with an error $\sigma_{space}$. 
\item Around the same time the instrument also points at an internal blackbody source at a measured temperature, defining a $counts_{BB}\left(T_{BB}\right)$ with errors of $\sigma_{sensor}$ and $\sigma_{placement}$ of that temperature sensor.
\item The aforementioned measured temperature is used to look up the radiance $R_{BB}\left(T_{BB}\right)$ for the given temperature in a previously determined calibration table.
\item By dividing this look-up radiance by the difference between the measured counts for space and black-body like so: $$ gain\left(T_{BB}\right) = \frac{-R_{BB}}{counts_{space} - counts_{BB}\left(T_{BB}\right)} $$ we define a gain for this blackbody temperature.
\item $$ \mathrm{Radiance} = \left(\mathrm{counts} - \mathrm{offset}\right) \cdot gain $$
\end{enumerate}

This is the general idea, while the devil is in the detail:
\begin{itemize}
\item Interpolation of HK blackbody temperature data. Interestingly, different time interval for telescope 1 and telescope 2.
\item above gain and offset are defined only defined for one calibration station, called "block" in this paper. They need to be smoothly interpolated so that every data sample has one offset and sample available.
\end{itemize}